\documentclass{report}
\usepackage{amsmath}
\usepackage{tcolorbox}

\begin{document}
	\title{Szabo and Ostlund}
	\author{Dhananjay}
	\maketitle
	\chapter{Mathematical Review}
	We start with discussing linear algebra by reviewing three-dimensional vector algebra. Starting from simple, then moving forward.
	\paragraph{Three-Dimensional Vector Algebra}
	
	A three dimensional vector can be represented by specifying its components $a_{i}$, i = 1,2,3,... with respect to a set of three mutually perpendicular unit vectors $\vec{\{e_i\}}$ as \\
	\begin{equation}\label{1.1}
		\vec{a} = \vec{e_1}a_1 + \vec{e_2}a_2 + \vec{e_3}a_3 = \sum_{i}\vec{e_i}a_i
	\end{equation}
	The vectors $\vec{e_i}$ are said to form a \textbf{basis}, and are called \textbf{basis vectors}. A basis is complete in a sense that any three-dimensional vector can be written as a linear combination of the basis vectors. However, a basis is not unique, and we could have chosen three different mutually perpendicular unit vectors, $\vec{\epsilon}_i$, i=1,2,3,... and represented $\vec{a}$ as 
	\begin{equation}\label{1.2}
		\vec{a} = \vec{\epsilon}_1a_1 + \vec{\epsilon}_2a_2 + \vec{\epsilon}_3a_3= \sum_{i}\vec{\epsilon}_ia_i
	\end{equation}
	Given a basis, a vector is completely specified by its three components with respect to that basis. Thus, we can represent the vector $\vec{a}$ by a column matrix as \\
	\begin{equation}\label{1.3}
			a = \begin{pmatrix}
			a_1 \\
			a_2 \\
			a_3 \\
		\end{pmatrix}
	\end{equation}
    in the basis $\vec{\{e_i\}}$, or as 
    	\begin{equation}\label{1.4}
    	a = \begin{pmatrix}
    		a_1 \\
    		a_2 \\
    		a_3 \\
    	\end{pmatrix}
    \end{equation}
	 in the basis $\vec{\{\epsilon\}}_i$
	 
	 The scalar product of two vectors $\vec{a}$ and $\vec{b}$ is defined as 
	 \begin{equation}\label{1.5}
	 	\vec{a}.\vec{b}=a_1b_1+a_2b_2+a_3b_3 = \sum_ia_ib_i
	 \end{equation}
	 Note that 
	 \begin{equation}\label{1.6}
	 \vec{a}.\vec{a} = a_1^2 + a_2^2 + a_3^2 \equiv |\vec{a}|^2
	\end{equation}
	 
	 is simply the square of the length (\vline$\vec{a}$\vline). Let us try and evaluate the scalar product $\vec{a}$.$\vec{b}$ using Eq. 1.1
	 
	\begin{equation}\label{1.7}
		\vec{a}.\vec{b}=\sum_i\sum_j\vec{e}_i.\vec{e}_ja_ib_j
	\end{equation}
	 For this to be identical to the definition of Eq. \ref{1.5}, we must have 
	\begin{equation}\label{1.8}
		\vec{e_i}\cdot\vec{e_j} = \delta_{ij} = \delta_{ji} =
		\begin{cases}
			1, & \text{if } i = j \\[6pt]
			0, & \text{otherwise}
		\end{cases}
	\end{equation}
	$\delta_{ij}$ means that the basis vectors are mutually perpendicular (orthogonal), and have unit length (normalized), in other words they are \textbf{orthonormal}.
		 Given a vector $\vec{a}$, we can find its component along $\vec{e_j}$ by taking the scalar product of Eq. \ref{1.1} with $\vec{e_j}$ and using the orthonormality relation (\ref*{1.8}). 
	\begin{tcolorbox}[colback=white, colframe=cyan, boxrule=1.5pt]
	 \begin{equation}
	{\vec{e_j}.\vec{a} = \sum_i\vec{e_j}.\vec{e_i}.a_i = \sum_i\delta_{ij}a_i = a_j}
\end{equation}
	\end{tcolorbox}
	

	\end{document}
